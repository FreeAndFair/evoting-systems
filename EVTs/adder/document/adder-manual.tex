\documentclass[letterpaper,10pt]{article}

%\usepackage[active]{srcltx}
\usepackage{graphicx}

\usepackage{fullpage}

\title{Using the Adder System}

\author{Eric Dutko \and Jason Grusauskas \and Michael Korman
\and David Walluck \and Aggelos Kiayias (faculty advisor)}

\begin{document}

\maketitle

\section{Setting up Adder}

Setting up Adder on the client system is a simple process. In the
most secure setting, the plugin would come pre-verified and
pre-installed on a dedicated voting machine. Internet voting,
however, requires an additional level of security.

\subsection{Plugin Installation}

This section is intended for Adder administrators only. Adder users
may skip to the next section.

If Internet voting is allowed from, say, a user's home computer
system, then the most secure way becomes offering a
cryptographically signed Mozilla installer. The Mozilla installer is
referred to as an XPI (cross-platform installer).

The XPI is signed with a Netscape tool know as \texttt{signtool}
which is part of the Netscape Security Suite (NSS).

In order to sign the installer, the system admin execute a command
similar to the following in the directory containing the plugin
files:

\begin{footnotesize}
\begin{verbatim}
signtool -d"/home/david/.netscape" -k"certs.adder.cse.uconn.edu/signing" \
  -p"password" .
\end{verbatim}
\end{footnotesize}

He may then check the result by executing

\begin{footnotesize}
\begin{verbatim}
signtool -v"/var/www/html/securevote/mozilla-plugin-adder.xpi"
\end{verbatim}
\end{footnotesize}

\begin{footnotesize}
\begin{verbatim}
using certificate directory: /home/david/.netscape
archive "/var/www/html/securevote/mozilla-plugin-adder.xpi" has passed crypto
verification.

          status   path
    ------------   -------------------
        verified   plugins/npadder.xpt
        verified   plugins/libnpadder.so
        verified   windows/npadder.dll
        verified   windows/npadder.xpt
        verified   install.js
\end{verbatim}
\end{footnotesize}

The installer contains a file inside it called
\texttt{META-INF/manifest.mf}, also used in JAR files.

A typical manifest entry inside the Mozilla XPI file (which is
really just a ZIP file containing a file called \texttt{install.js}
looks like

\begin{footnotesize}
\begin{verbatim}
Name: plugins/npadder.xpt
Digest-Algorithms: MD5 SHA1
MD5-Digest: XiANyrohHSEeN07goQZnQQ==
SHA1-Digest: e20TrROsMC7GyafACvllkfB+9eE=
\end{verbatim}
\end{footnotesize}

and contains both the MD5 and SHA1 hash for each file contained
within the archive.

\subsection{Installing the Adder CA Certificate}

This section deals with the system user who is sitting at his web
browser. The Mozilla plugin supports both Windows and Unix systems.

The plugin has two modes, signed and unsigned. If the install XPI
is unsigned, you can skip this section.

The install XPI can be cryptographically signed with an MD5 or SHA1
hash. This verifies the integrity of the installer. The current
certificate fingerprints are listed below.

\begin{footnotesize}
\begin{verbatim}
SHA1 Fingerprint: 0C:D7:02:46:48:08:9F:D4:E3:8B:6C:EF:06:11:FF:C2:44:E5:40:34
MD5 Fingerprint:  75:14:93:47:5A:62:B5:EC:03:5A:EC:4B:FA:76:71:84
\end{verbatim}
\end{footnotesize}

A user can click `View' to examine the certificate fingerprint and
verify that the certificate is correct. Of course, security is
subverted if the user does not bother to check the origin of the
certificate. This is yet an additional problem for Internet voting
which is separate from the Adder system itself.

When the user first accesses the Adder website, he will see a status
message indicating that he should install the certificate if he
hasn't already. Once you click the link you should get a box as
follows:

\includegraphics*[width=\textwidth]{img/cert}

You should select at least `Identify Software Developers'.
Selecting `Identify websites' will allow the same certificate to be
used for the Adder website as well.

\subsection{Installing the Adder Plugin}

Click the appropriate link to begin the installation. When you do,
you will be presented with a box as follows:

\includegraphics*[width=\textwidth]{img/inst1}

If the XPI installer is signed, you will see
\textbf{certs.adder.cse.uconn.edu}. Otherwise, you will see
\textbf{Unsigned}.

After clicking OK, if all goes well, you should see a box as
follows:

\includegraphics*[width=\textwidth]{img/inst2}

After a successful installation, you will need to restart your
browser. Incidently, the plugin (usually with extension \texttt{.so}
on Unix, and \texttt{.dll} on Windows) is able to be used without
restarting, but the scriptable component (with extension
\texttt{.xpt} required by the plugin will not be usable until after
a browser restart.

\subsection{Verifying Plugin Installation}

Upon restarting, the statusbar at the bottom of the page should say
`Plugin integrity check: passed'. The integrity check makes sure
that all scriptable functions are available. It should look
something like:

\includegraphics*[width=\textwidth]{img/stat1}

Also, going to \texttt{about:plugins} in the location bar should
show the plugin is installed.

\includegraphics*[width=\textwidth]{img/about}

\section{Using Adder}

You can use adder in two ways, either as an authority, or as a
normal user (voter).Adder has functionality which is shared between
authorities and users.

\subsubsection{Secure Logout}

The first of this functionality is secure logout. Adder has the
ability to end your session and log you out of the adder system.
Note that Adder cannot control whether or not you have saved your
password in your browser. For maximum, security, it is recommended
that you do not save your password, particularly if you are an
authority.

\includegraphics*[width=\textwidth]{img/00-logout}

\subsubsection{Procedure Selection}

The initial page lists a set of currently active voting procedures.
Here a user is able to either log-in to a procedure, or view the
current procedure status.

Any active procedures are listed when you log in. If there are no
active procedures, the page will be empty.

\includegraphics*[width=\textwidth]{img/01-procedure-selection}

\subsubsection{Logging In}

When presented with the login page, the user will enter his name and
password. The system knows which accounts are authority accounts and
which account are normal user accounts. Based on the state of the
system, the authority will either be able to submit his key, or
enter the voting process. A normal user will either be able to vote
only if the authorities have completed what is required to set up a
voting procedure.

\includegraphics*[width=\textwidth]{img/02-login1}

\subsubsection{Entering your password}

To login, enter your login name and password into their respective
boxes.

\includegraphics*[width=\textwidth]{img/03-login2}

\subsection{Using Adder as an Authority}

Authorities must initiative an election before users may vote.
Initiating a voting procedure takes multiple steps.

\subsubsection{Creating a key pair}

The first step is to create a keypair. This key pair will be used to
encrypt the authorities' polynomials. It is crucial that you do not
allow other users to access this directory.

\includegraphics*[width=\textwidth]{img/04-create-keypair1}

When the key is created successfully, a dialog box will appear. At
this stage you will have files with the \texttt{.adder} extension in
a directory called \texttt{.adder} which is created under your
\texttt{\$HOME} directory on Unix systems. On Windows, the location
may vary. On multiuser systems, such as Windows NT, Windows 2000,
and Windows XP, the environmental variable \texttt{USERPROFILE} is
used. On Windows 9x systems, the Windows directory is used. As a
last resort, the \texttt{/} directory is used. Note that as an
authority, you should not be using a Windows 9x system which does
not provide file protection via permissions or access lists.

\includegraphics*[width=\textwidth]{img/05-create-keypair2}

After creating the keypair, click the second button that will submit
the key to the server. A success page should appear.

\includegraphics*[width=\textwidth]{img/06-key-success}

At this point, you may view the server status.

\includegraphics*[width=\textwidth]{img/07-status}

If any other authorities are waiting to submit keys, you will see
the following stage.

\includegraphics*[width=\textwidth]{img/08-status-waitkeys}

\subsection{Submitting the polynomial}

At this point you may submit the polynomial, encrypted with your
key, to the server.

\includegraphics*[width=\textwidth]{img/09-generate-polynomial}

If successful, you will see the following page.

\includegraphics*[width=\textwidth]{img/10-polynomial-success}

\textbf{This happens when you log in, why is the screenshot here?}

\includegraphics*[width=\textwidth]{img/11-save-private-key}

\subsection{Accepting votes}

At this point the server should be ready to accept votes. The status
page should appear as shown below.

\includegraphics*[width=\textwidth]{img/12-status-accepting-votes}

\subsection{Using Adder as a Voter}

\subsubsection{Placing a vote}

Voting is what the user will be most interesting in. After logging
in to a procedure, the user will be presented with a page that
contains a description or question followed by a list on possible
choices. Next to each choice is a radio button. By default, no
choices should be selected.

\includegraphics*[width=\textwidth]{img/13-voting1}

\subsubsection{Confirming your vote}

Before submitting the vote, you should be sure that the system
agrees with the vote that you think you have submitted. A dialog box
will appear that will show your current choice, as well as what the
choice looks like when encrypted. You are able to write this down
and verify later that the system has indeed recorded the correct
vote for you.

\includegraphics*[width=\textwidth]{img/14-voting2}

\subsubsection{Submitting the vote}

If your vote is incorrect, you may pick \texttt{Cancel} and choose
again. Otherwise, click \texttt{OK} to submit your vote.

After submitting, you should see a success page like the one shown
below.

\includegraphics*[width=\textwidth]{img/15-vote-success}

\subsubsection{Checking voting status}

If you check the server status during voting, you will see a page
like the one shown below. On this page you will see a list of users
who have voted as well as a icon next to their vote. Click the icon
to see or verify the encrypted vote.

\includegraphics*[width=\textwidth]{img/16-status-accepting-votes2}

\subsubsection{Verifying a vote}

When the icon is clicked, you will see a vote that looks like the
one shown below. Every vote is unique. An adversary will not be able
to tell who or what you have voted for, but you, on the other hand,
will be able to check that the system has correctly recorded your
vote.

\includegraphics*[width=\textwidth]{img/17-show-vote}

\subsection{Decrypting the voting results}

After the voting session has ended the authorities must decrypt the
vote. The authority will see a page like the one shown below after
he has logged in.

\includegraphics*[width=\textwidth]{img/18-decrypt}

If the decryption was successful, you will see a success page like
the one shown.

\includegraphics*[width=\textwidth]{img/19-decrypt-success}

\subsubsection{Checking the decryption status}

During this stage, you can still check users votes. The server will
let you know that it is waiting for authorities in order to be able
to retrieve the results of the election.

\includegraphics*[width=\textwidth]{img/20-status-waitdecrypt}

After the election has ended, the status will change to let you
know, but all other functionality still remains available during
this time.

\includegraphics*[width=\textwidth]{img/21-status-done}

\subsection{Viewing the results}

In the final stage you will see the final votes of the voting
procedure. This page may vary depending on the type of vote held.

\includegraphics*[width=\textwidth]{img/22-results}

\end{document}
